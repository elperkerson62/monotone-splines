\documentclass{article}

\usepackage{
    graphicx,
    fullpage,
    amsmath,
    amssymb,
    dsfont,
    url,
    amsthm
}

\usepackage[parfill]{parskip}
\usepackage[ruled]{algorithm2e}

\DeclareMathOperator*{\argmin}{argmin}
\DeclareMathOperator{\prox}{prox}
\DeclareMathOperator{\id}{Id}
\DeclareMathOperator{\svd}{svd}
\newtheorem{theorem}{Theorem}
\newtheorem{lemma}{Lemma}
\newtheorem{definition}{Definition}
\newtheorem{corollary}{Corollary}


\newcommand*{\vertbar}{\rule[-1ex]{0.5pt}{2.5ex}}
\newcommand{\header}[5]{
    \begin{center}
    {
        \small  #1 #2
        \hfill { {Date: #5} }\\ \vspace{.5pt}
        #3 \hfill
        {
             
        }
    }
    \vspace{-1ex}
    \hrulefill\\
    \vspace{4ex}
    {
        \LARGE Monotone Splines for Regression
    } \\
    \vspace{20pt}
    \end{center}
}


\begin{document}
\pagenumbering{gobble}

\header{}{Eric Perkerson}{}{1}{March 23, 2020}
\begin{center}
{ \large }
\end{center}

\section{The Regression Problem}

Given some statistical data $\{ (x_i, y_i) \}_{i = 1}^n$ where $x_i$, $y_i \in \mathbb{R}$, we the regression problem is the problem of estimating the function $f(x) = E(y | x)$ where $f \colon \mathbb{R} \to \mathbb{R}$. The primary difficulty here is to avoid overfitting, that is, to construct an estimate $\widehat{f}$ for the unkown function $f$ that is close in some sense, not merely one that is close for the observed data points. This is the generalization problem. A good estimate $\widehat{f}$ is close to $f$ even for non-observed values of $x$. To find such an estimate, we have make some prior assumptions about $\hat{f}$. These are typically enforced by either restricting the function class $V$ from which we choose $\widehat{f}$ or by penalizing certain functions. Linear regression is an example of an extremely strong limitation on the class of functions used, viz. $V = \{ f \colon f(x) = ax + b\}$. Here we use monotone splines (M-splines), integrated splines (I-splines), and convex splines (C-splines) as ways of restricting the class of functions to avoid overfitting.

\section{Monotone Splines}

We first a degree $k$ for the spline function $\widehat{f}$ and a closed interval $[a, b]$ for the domain of $\widehat{f}$. This can be done, for instance, by using the minimum and maximum value of the set $\{ x_i \}$. We then choose $m$ interior break points $t_{k + 1}, \dots, t_{k + m} \in (a, b)$. Lastly, define $k$ additional break points at each end point $t_1 = \dots = t_k = a$ and $t_{m + k + 1} = \dots = t_{m + 2k} = b$ for notational simplicity.

The monotone splines or M-splines of order $k$ are given by the recursive formula
\begin{align*}
    M^{(1)}_i (x) & = \frac{\mathds{1}_{[t_i, t_{i+1}]}}{t_{i+1} - t_i} \text{ if $t_{i+1} \ne t_i$ } \\
    M^{(1)}_i (x) & = 0 \text{ if $t_{i+1} = t_i$ } \\
    M^{(k)}_i (x) & = \frac{k[(x - t_i)M^{(k - 1)}_i (x) + (t_{i + k} - x)M^{(k - 1)}_{i+1} (x)]}{(k-1)} \\
\end{align*}
and then integrated and convex splines are defined by
\begin{align*}
    I^{(k)}_i (x) & = \int_a^x M^{(k)}_i (t) dt \\
    C^{(k)}_i (x) & = \int_a^x I^{(k)}_i (t) dt
\end{align*}

\section{Degree $k = 1$}

$t$

\section{Degree $k = 2$}

For $i = 1$:
\[
    M^{(2)}_1 (x) = \begin{cases}
        0 ,&  x \notin [t_i, t_{i+2}] \\
        \frac{2(x - t_i)}{(t_{i+2} - t_i)(t_{i+1} - t_i)} ,& x \in [t_i, t_{i+1}] \\
        \frac{-2(x - t_{i+2})}{(t_{i+2} - t_i)(t_{i+2} - t_{i+1})} ,& x \in [t_{i+1}, t_{i+2}] \\
    \end{cases}
\]


For $i = 2, \dots, k + m - 1$: 
\[
    M^{(2)}_i (x) = \begin{cases}
        0 ,& x \notin [t_i, t_{i+2}] \\
        \frac{2(x - t_i)}{(t_{i+2} - t_i)(t_{i+1} - t_i)} ,& x \in [t_i, t_{i+1}] \\
        \frac{-2(x - t_{i+2})}{(t_{i+2} - t_i)(t_{i+2} - t_{i+1})} ,& x \in [t_{i+1}, t_{i+2}] \\
    \end{cases}
\]

\section{Degree $k = 3$}




\end{document}